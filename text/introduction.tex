    За последнее десятилетие сложность и способности нейронных сетей существенно выросли, однако их потенциал до сих пор ограничивают стоимость и энергия потребления. Как известно, нейронные сети состоят из нескольких слоев взвешенных сумм, которые предсказывают нужный результат. Хранение всех значений чисел с плавающей запятой значительно увеличивает время обучения и требует большое количество памяти, что вызывает необходимость использовать только специальное оборудование, которое выдержит подобную нагрузку.
    
    Чтобы устройство с ограниченными ресурсами могло решать такие проблемы глубокого обучения, как распознавание лиц в реальном времени, необходимо использовать в качестве весов бинарные числа, а в качестве функций активации "--- бинарные аналоги функций активации в непрерывном случае, то есть <<бинаризовать>> нейронную сеть. Это позволит хранить гораздо больший объем данных, используя, например, 32-битный контроллер. Использование битовых операций сокращает время исполнения. Размеры бинарных нейронных сетей намного меньше, чем у их вещественнных аналогов. Точность моделей также меньше, но эта разница в точности постепенно сокращается и бинарные нейронный сети становятся точнее на больших датасетах, как ImageNet.
    
    В данной работе рассматривается вопрос полноты бинарной нейронной сети, то есть способность в точности выразить булеву функцию. А также ставится задача аппроксимации булевых функций, то есть способность выразить булеву функцию с некоторой допустимой погрешностью. С практической точки зрения устраивают оба варианта.