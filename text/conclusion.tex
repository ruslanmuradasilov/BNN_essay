    В данной работе исследована проблема полноты бинарной нейронной сети. Доказывается, что для полноты бинарных нейронных сетей, слоем которой является композиция линейной булевой функции и нелинейного бинарного оператора, в некоторых случаях достаточно двух слоев, а  в общем случае "--- четырёх. В дальнейшем планируется получить результаты по глубине для разных функций активации.
    
    Обозначена важность проблемы аппроксимации булевых функций линейными функциями.
    Предложена идея доказательства плохой аппроксимации линейными функциями. Ставится цель изучить аппроксимацию в бинарном случае нейронной сетью, в слое которой будут линейные функции и некоторые нелинейности, а также применять подобную архитектуру в практических задачах.
    
    Автор выражает благодарность И. Е. Иванову за научное руководство.