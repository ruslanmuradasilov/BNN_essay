\begin{thebibliography}{99}

\bibitem{first} Courbariaux, M.; Bengio, Y. BinaryNet: Training Deep Neural Networks with Weights and Activations Constrained to $+1$ or $-1$. arXiv:1602.02830.
    
\bibitem{second} Rastegari, M.; Ordonez, V.; Redmon, J.; Farhadi, A. XNOR-Net: ImageNet Classification Using Binary Convolutional Neural Networks. In Proceedings of the European Conference on Computer Vision, Amsterdam, The Netherlands, 11–14 October 2016; pp. 525–542.32.
    
\bibitem{third} Zhou, S.; Ni, Z.; Zhou, X.; Wen, H.; Wu, Y.; Zou, Y. DoReFa-Net: Training Low Bitwidth Convolutional Neural Networks with Low Bitwidth Gradients. arXiv 2016, arXiv:1606.06160.
    
\bibitem{fourth} Tang, W.; Hua, G.; Wang, L. How to Train a Compact Binary Neural Network with High Accuracy? In Proceedings of the Thirty-First AAAI Conference on Artificial Intelligence, San Francisco, CA, USA, 4–9 February 2017
    
\bibitem{fifth} Darabi, S.; Belbahri, M.; Partovi Nia, V.; Courbariaux, M. Regularized Binary Network Training. arXiv:1812.11800
    
\bibitem{sixth} Simons, T.; Lee, D. A Review of Binarized Neural Networks. Electrical and Computer Engineering, Brigham Young University, Provo, UT 84602, USA.
    
\bibitem{seventh} Колмогоров А. Н. О представлении непрерывных функций нескольких переменных суперпозициями непрерывных функций меньшего числа переменных. ДАН СССР, 1956, Т.108, №2, С. 179-182
    
\bibitem{eighth} Арнольд В.И. О функции трех переменных. ДАН СССР, 1957, Т.114, №4, С. 679-681
    
\bibitem{ninth} Колмогоров А. Н. О представлении непрерывных функций нескольких переменных в виде суперпозиций непрерывных функций одного переменного и сложения. ДАН СССР, 1957, Т.114, №5, С. 953-956
    
\bibitem{tenth} Hecht-Nielsen R. Kolmogorov’s mapping neural network existence theorem. IEEE First Annual Int. Conf. on Neural Networks, San Diego, 1987. Vol. 3. — P. 11—13.

\bibitem{eleventh} Алексеев В. Б. Дискретная математика (II семестр). — М., МГУ, 2002. — 44 с.

\bibitem{twelfth} Cybenko, G. V. Approximation by Superpositions of a Sigmoidal function // Mathematics of Control Signals and Systems. — 1989. — Т. 2, № 4. — С. 303—314.

\end{thebibliography}